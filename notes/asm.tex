\documentclass[a4paper, 12pt]{article}

\title{x86-64 assembly programming}
\author{P C Joby}
\date{23 Oct, 2014}

\begin{document}
\maketitle

\section{Computer Memory}

Hardware mapping registers on an \textit{x86-64} can map pages of 
2 different sizes - 4 KiB and 2 MiB. \textit{Linux} uses 2 MiB
for kernel and 4 KiB pages for most other uses.

\section{Process memory model in Linux}

In Linux, memory of a process is divided into 4 logical regions: text,
data, heap and stack. The stack is mapped to the highest address of a 
process and on x86-64 Linux this is \texttt{0x7fff\_ffff\_ffff\_ffff\_ffff\_ffff} or 131 TiB.

The lowest address in x86-64 process (text segment) is \texttt{0x40\_0000} (4 MiB).
The stack segment is restricted in size to typically 16 MiB.

\section{Memory mapping in 64-bit mode}

CR3 register holds the address of PML4 (Page Map Level 4) table.
Of the 64 bits of virtual address \textit{63-48} are \textit{unused}.
Bits \textit{47-39} are used as an index into PML4 table. PML4 table is 
essentially an array of 512 pointers. These pointers point to pages of memory
so the rightmost 12 bits of each pointer can be used for other purposes like
indicating whether an entry is valid or not. Generally not all entries in PML4 table would be valid.

The next level in memory translation hierarchy is the collection of page directory pointer tables. Each of these tables is also an array of 512 pointers.
Bits \textit{38-30} are used as index into PDPT.

The third level in memory translation heirarchy is the collection of page directory tables. Bits \textit{29-21} are used as an index into PDT.

The fourth and last level in memory translation heirarchy is the collection
of page tables. Again each of this table is a collection of 512 pointers. Bits \textit{20-12} are used as an index into PT. 

After using 4 tables we reach the address of the page of memory which was 
originally referenced. Then we can add the page offset (bits 11-0) to get the final physical address.

CPU designers have added support for large pages using 3 levels of existing
translation tables. By using 3 levels of tables, there are 9 + 12 = 21 bits left for the page offset field. This make large pages \(2^{21}\) bytes (2 MiB). 

 
\end{document}
